\documentclass[12pt,a4paper]{article}
\usepackage[utf8]{inputenc}
\usepackage[T1]{fontenc}
\usepackage{lmodern}
\usepackage{amsmath, amssymb, amsthm, amsfonts, graphicx, physics, enumitem, geometry, esint}
\geometry{margin=1in}
\usepackage[dvipsnames]{xcolor}
\definecolor{EvanPink}{rgb}{0.85, 0.13, 0.55}
\usepackage[colorlinks=true,
            linkcolor=OliveGreen,
            urlcolor=EvanPink,
            citecolor=OliveGreen]{hyperref}
\usepackage{cleveref}
\usepackage{etoolbox}
\newenvironment{solution}{%
  \par\noindent\textit{Solution.}\ }{\qed}
\newtheoremstyle{problemstyle}
  {1em}   
  {1em}   
  {}      
  {}      
  {\bfseries} 
  {.}     
  {0.5em} 
  {}      
\theoremstyle{problemstyle}
\newtheorem{problem}{Problem}[section]
\apptocmd{\problem}{%
  \addcontentsline{toc}{subsection}{Problem~\thesection.\arabic{problem}}%
}{}{}
\crefname{problem}{Problem}{Problems}
\Crefname{problem}{Problem}{Problems}
\setcounter{tocdepth}{2}
\title{\textbf{Chapter 2 : Integration}}
\author{Arkaraj Mukherjee}
\def\bN{\mathbb{N}}
\def\bR{\mathbb{R}}
\def\bZ{\mathbb{Z}}
\def\bQ{\mathbb{Q}}
\def\bC{\mathbb{C}}
\def\mM{\mathcal{M}}
\def\mR{\mathcal{R}}
\def\mN{\mathcal{N}}
\def\mE{\mathcal{E}}
\def\mA{\mathcal{A}}
\def\oR{\overline{\mathbb{R}}}
\begin{document}
\maketitle
\newpage
\tableofcontents
\newpage
\section{Chapter 2 : Integration}
\begin{problem}\label{prob:2.1}
Let $f: X \to \overline{\bR}$ and $Y = f^{-1}(\bR)$. Then $f$ is measurable iff $f^{-1}(\{-\infty\}) \in \mM$, $f^{-1}(\{\infty\}) \in \mM$, and $f|_Y$ is measurable on $Y$.
\end{problem}
\begin{solution}
    For the only if direction, $\{\infty\},\{-\infty\},\bR=Y\in\mathcal B_{\oR}$ making $f^{-1}$ map them into $\mM$ as $f$ is measurable and as $\mM$ is a sigma algebra we see that $f\vert_Y:Y\to\oR$ is $\mM_Y$-measurable as $f^{-1}(E)\cap Y\in\mM_Y$ for all $E\in\mathcal B_{\oR}$. 
    Now for the if direction, as $\mathcal B_{\oR}=\{E\cup I:E\in\mathcal B_{\bR}, I\subseteq\{\infty,-\infty\}\}$ we see that for any such $E\cup I\in\mathcal B_{\oR}$ we have $f^{-1}(E\cup I)=f^{-1}(E)\cup f^{-1}(I)$ and as $f^{-1}(E)\cap Y=f^{-1}(E)\in\mM_Y\subseteq\mM$ from and $f^{-1}(I)\in\mM$ so $f$ is measurable on $\mM$.
\end{solution}
\begin{problem}\label{prob:2.2}
Suppose $f, g: X \to \oR$ are measurable.
\begin{enumerate}[label=\alph*.]
    \item $fg$ is measurable (where $0 \cdot (\pm\infty) = 0$).
    \item Fix $\alpha \in \oR$ and define $h(x) = \alpha$ if $f(x) = -g(x) = \pm\infty$ and $h(x) = f(x) + g(x)$ otherwise. Then $h$ is measurable.
\end{enumerate}
\end{problem}
\begin{solution}
    \begin{enumerate}[label=\alph*.]
        \item Consider the two maps,
        $$X\ni x\longrightarrow (f(x),g(x))\text{ and }\oR^2\ni(x,y)\mapsto xy$$
        the first is a measurable map from $X$ (with sigma algebra $\mM$) to $R\times R$ (with sigma algebra $\otimes^2_1 \mathcal B_{\oR}$) and the second from this to $\oR$ (with sigma algebra $\mathcal B_{\oR}$) and as the composition of measurable maps are measurable we are done.
        Here, the measurability of $(x,y)\to xy$ follows from its continuity (note: the metric on $\oR$ considered throughout the chapter is usually $d(x,y):=|\arctan(x)-\arctan(y)|$ with $\arctan(\pm\infty)=\pm\pi/2$).
        \item Similarly again consider $x\mapsto(f(x),g(x))$ but this time with $(x,y)\to J(x,y)$ where,
        $$J(x,y):=\begin{cases}\alpha\text{ if }x=-y=\pm\infty\\ x+y\text{ otherwise}\end{cases}$$
        We will use \hyperref[prob:2.5]{Problem 1.5} here. 
        Let $A=\{(+\infty,-\infty),(-\infty,+\infty)\}$ and $B=\oR^2\setminus A$, it is clear that $J$ is continuous on $B$ and hence measurable on it and as it is constant on $A$ it is trivially measurable on it so we are done.
        As compositions of measurable functions are measurable we are done.
    \end{enumerate}
\end{solution}
\begin{problem}\label{prob:2.3}
If $\{f_n\}$ is a sequence of measurable functions on $X$, then $\{x : \lim f_n(x) \text{ exists}\}$ is a measurable set.
\end{problem}
\begin{solution}
    We know that $g_1=\limsup f_n$ and $g_2=-\liminf f_n$ are measurable functions, define 
    $$h(x):=\begin{cases}2026\text{ if }g_1(x)=-g_2(x)=\pm\infty\\ g_1(x)+g_2(x)\text{ otherwise}\end{cases}$$
    $h$ is a measurable function by \hyperref[prob:2.2]{Problem 1.2} and $f_n$ converges iff $g_1=-g_2$ and we get that 
    $$\left\{x:\lim f_n(x)\text{ exists}\right\}=h^{-1}(\{0,2026\})$$
    which implies that the set infact is measurable as all the sets we performed unions and intersections on are in $\mM$.
\end{solution}
\begin{problem}\label{prob:2.4}
If $f: X \to \overline{\bR}$ and $f^{-1}((r, \infty]) \in \mM$ for each $r \in \bQ$, then $f$ is measurable.
\end{problem}
\begin{solution}
    It can easily be shown that these sets being in $\mM$ implies that $f^{-1}((k,\infty])\in\mM$ for any $k\in\oR$ easily so we are done.
\end{solution}
\begin{problem}\label{prob:2.5}
If $X = A \cup B$ where $A, B \in \mM$, a function $f$ on $X$ is measurable iff $f|_A$ is measurable on $A$ and $f|_B$ is measurable on $B$.
\end{problem}
\begin{solution}
    If $f$ is measurable then measurability on $A,B\in\mM$ are trivial.
    For the other direction, suppsoe $f:X\to Y$ with $\mN$ being the sigma algebra on $Y$, for any $E\in\mN$ we have,
    $$f^{-1}(E)=(f^{-1}(E)\cap A)\cup(f^{-1}(E)\cap B)=(f\vert_A^{-1}(E))\cup(f\vert_B^{-1}(E))\in\mM$$
    as these two sets are in $\mM_A,\mM_B\subseteq\mM$ respectively anyways.
\end{solution}
\begin{problem}\label{prob:2.6}
The supremum of an uncountable family of measurable $\bR$-valued functions on $X$ can fail to be measurable (unless the $\sigma$-algebra $\mM$ is very special).
\end{problem}
\begin{solution}
    Suppose $X=\bR$ with the sigma algebra being the lebesgue measurable sets, then for a non measurable set $E\subseteq\bR$ consider the family $\{\chi_{\{r\}}\}_{r\in E}$, these are all measurable clearly whereas $\sup_{r\in E}\chi_{\{r\}}=\chi_E$ is clearly non measurable unless $\mM$.
\end{solution}
\begin{problem}\label{prob:2.7}
Suppose that for each $\alpha \in \bR$ we are given a set $E_\alpha \in \mM$ such that $E_\alpha \subset E_\beta$ whenever $\alpha < \beta$, $\bigcup_{\alpha \in \bR} E_\alpha = X$, and $\bigcap_{\alpha \in \bR} E_\alpha = \varnothing$. Then there is a measurable function $f: X \to \bR$ such that $f(x) \le \alpha$ on $E_\alpha$ and $f(x) \ge \alpha$ on $E_\alpha^c$ for every $\alpha$. (Use Exercise 4.)
\end{problem}
\begin{solution}
    We claim that the following function works
    $$f(x) := \sup\{r\in\bQ:x\notin E_r\}$$
    For $x\in E_\alpha$, from the definition we have that $x\in E_\beta$ for all $\bQ\ni\beta\geq\alpha$ and thus $f(x)=\sup\{r\in\bQ:r<\alpha, x\notin E_r\}\leq\alpha$. 
    Now, for $x\in E_\alpha^c$ its clearly $\geq\alpha$. 
    We claim that, for $r\in\bQ$ we have 
    $$f^{-1}((r,\infty))\bigcup_{\bQ\ni\alpha\geq r}E_\alpha^c$$
    If $f(x)>r$, then the set of $\beta$ for which $x\in E^c_\beta$ (its an open ray unbounded below) must contain $r$, i.e its of form $(-\infty,v)$ with $v>r$ and we can find a rational $q\in(r,v)$ such that $x\in E_q$ so we have one direction of the equality, the other direction is trivial and we are done as the set on the right is in $\mM$ being a countable union of sets from $\mM$ which is sufficient to prove measurablity via \hyperref[prob:2.4]{Problem 2.4}
\end{solution}
\begin{problem}\label{prob:2.8}
If $f: \bR \to \bR$ is monotone, then $f$ is Borel measurable.
\end{problem}
\begin{solution}
    trivial.
\end{solution}
\begin{problem}\label{prob:2.9}
Let $f: [0, 1] \to [0, 1]$ be the Cantor function ($\S 1.5$), and let $g(x) = f(x) + x$.
\begin{enumerate}[label=\alph*.]
    \item $g$ is a bijection from $[0, 1]$ to $[0, 2]$, and $h = g^{-1}$ is continuous from $[0, 2]$ to $[0, 1]$.
    \item If $C$ is the Cantor set, $m(g(C)) = 1$.
    \item By Exercise 29 of Chapter 1, $g(C)$ contains a Lebesgue nonmeasurable set $A$. Let $B = g^{-1}(A)$. Then $B$ is Lebesgue measurable but not Borel.
    \item There exist a Lebesgue measurable function $F$ and a continuous function $G$ on $\bR$ such that $F \circ G$ is not Lebesgue measurable.
\end{enumerate}
\end{problem}
\begin{solution}
\begin{enumerate}[label=\alph*.]
    \item It is known that the cantor function is continuous on $[0,1]$ as well as being increasing. 
    This makes $g(x)=f(x)+x$ strictly increasing and continuous, with this we can say using the intermediate value theorem that $g$ is is a continuous bijection from $[0,1]$ to $[0,2]$ and $h=g^-1$ is also continuous. 
    \item As the cantor set is borel and $h$ is continuous, it must be borel measurable. 
    Thus, $h^{-1}(C)=g(C)$ is borel (as well as being lebesgue measurable which is weaker). 
    We can write $m(g([0,1]))=m(g(C))+m(g([0,1]\setminus C))$ as the sets here are disjoint (the second one is clearly measurable). 
    From the previous parts we see that $m(g(C))=2-m(g([0,1]\setminus C))$. 
    This set is the disjoint union of the middle third intervals removed in the process of constructing the cantor set, on which, by defintion the cantor function is constant. 
    These removed open intervals have measures summing to $1$ and as $g$ is a translation on these we see that $m(g([0,1]\setminus C))=1$ showing that $m(g(C))=1$. 
    \item As the lebesge measure is complete we see that $B$ being the subset of the cantor set which has measure zero, must be lebesgue measurable too. 
    Now, if $B$ was borel then so would be $h^-1(B)=A$ as $g^{-1}$ is continuous, so it is not borel. 
    \item Consider $\chi_{B}\circ h$, we see that $(\chi_B\circ h)^{-1}(\{1\})=h^{-1}(\chi_B^{-1}(\{1\}))=h^{-1}(B)=A$ which is not lebesgue measurable while we had $h$ be continuous (a homomorphism even) and $\chi_B$ lebesgue measurable so we are done as $\{1\}$ is borel.

\end{enumerate}
\end{solution} 

\begin{problem}\label{prob:2.10}
Prove that the following implications are valid iff the measure $\mu$ is complete:
\begin{enumerate}[label=\alph*.]
    \item If $f$ is measurable and $f=g$ $\mu$-a.e. then $g$ is measurable.
    \item If $f_n$ is measurable for $n\in\bN$ and $f_n\longrightarrow f$ $\mu$-a.e. then $f$ is measurable. 
\end{enumerate}
\end{problem}
\begin{solution}
    \begin{enumerate}[label=\alph*.]
        \item First assume that $\mu$ is complete and that $f$ and $g$ disagree on some $N\in\mM$ with $\mu(N)=0$, then for any measurable $E$ in the range we must have that 
        $$g^{-1}(E)=(g^{-1}(E)\cap N^c)\cup(g^{-1}(E)\cap N)=(g|_{N^c}^{-1}(E))\cup(g^{-1}(E)\cap N)$$
        $$=(\underbrace{f|_{N^c(\in\mM)}^{-1}(E)}_{\in\mM})\cup(\underbrace{g^{-1}(E)\cap N}_{\subseteq N\implies\in\mM})$$
        and thus this set must be in $\mM$ itself. 
        For the other direction, suppose $\mu(N)=0$ for some $N\in\mM$ and $K\subseteq N$. 
        Let, $f=\chi_N$ and $g=\chi_K$, we see that $f=g$ on $N^c$ and thus $f=g$ $\mu$-a.e. which makes $g$ measurable and thus $g^{-1}(\{1\})=K$ must also be measurable making $\mu$ complete. 
        \item For one direction assume that $\mu$ is complete. 
        By \hyperref[prob:2.3]{Problem 2.3} we know that the set where we have convergence, is measurable so let it be $E$. 
        Further, a.e. convergence implies that $\mu(X\setminus E)=0$. 
        Let $g=\lim\chi_E f_n(x)$, we see that $g$ is measurable on $E$ as well as on $E^c$ and is thus measurable on $X$, being the limit (it always exists) of measurable functions. 
        Suppose, $N$ is the set where $f_n\not\longrightarrow f$, by our hypothesis this is of measure zero. 
        For $x\in N^c$ however $f_n(x)\longrightarrow f(x)$ and thus $f(x)=g(x)$ so $f(x)\neq g(x)$ can only happen on $N$ implying that $f=g$ a.e after which part a proves measurablity of $f$ and we are done. 
        For the other direction, suppose $E$ is a set of measure zero and $K\subseteq E$. 
        Let, $f=\chi_K$ and $f_n=0$, with this we see that $f_n\to f$ on $E^c$ which shows that $f_n\to f$ a.e. and thus $f$ is measurable so we are done. 
    \end{enumerate}
\end{solution}
\begin{problem}\label{prob:2.11}
Suppose that $f$ is a function on $\bR \times \bR^k$ such that $f(x, \cdot)$ is Borel measurable for each $x \in \bR$ and $f(\cdot, y)$ is continuous for each $y \in \bR^k$. For $n \in \bN$, define $f_n$ as follows. For $i \in \bZ$ let $a_i = i/n$, and for $a_i \le x \le a_{i+1}$ let
\[ f_n(x, y) = \frac{f(a_{i+1}, y)(x - a_i) - f(a_i, y)(x - a_{i+1})}{a_{i+1} - a_i} \]
Then $f_n$ is Borel measurable on $\bR \times \bR^k$ and $f_n \to f$ pointwise; hence $f$ is Borel measurable on $\bR \times \bR^k$. Conclude by induction that every function on $\bR^n$ that is continuous in each variable separately is Borel measurable.
\end{problem}
\begin{solution}
    We can write $\bR\times\bR^k$ as a countable union of the borel sets $\{[a_i,a_{i+1}]\times\bR^k\}_{i\in\bZ}$ and proving borel measurablity of $f_n$ on these suffices to prove borel measurablity globally. 
    It being borel on $[a_i,a_{i+1}]\times\bR^k$ follows from these maps being clearly borel 
    $$(x,y)\mapsto (x-a_i),(x,y)\mapsto (x-a_{i+1}),(x,y)\mapsto f(a_i,y),(x,y)\mapsto f(a_{i+1},y)$$ 
    as the borel measurability is preserved under products and sums. 
    To prove pointwise convergence just notice that for $(x,y)\in[a_i,a_{i+1}]\times\bR^k,f_n(x,y)$ is between $f(a_i,y)$ and $f(a_{i+1},y)$ which converge to $f(x,y)$ as $n\to\infty$ by continuity of $f(\cdot,y)$. 
    This shows that if $f(\cdot,y)$ is continuous and $f(x,\cdot)$ borel, for all $(x,y)\in\bR\times\bR^k$ then $f$ is borel. 
    Now, suppose that $f:\bR^n\to\bR$ is continuous in each variable, suppose that the proposition holds for $n-1$, then for all $x\in\bR,f(x,\cdot):y\in\bR^{n-1}\mapsto f(x,y)$ is borel because this is continuous in each variable as well and for all $y\in\bR^{n-1},f(\cdot,y):x\in\bR\mapsto f(x,y)$ is continuous too, which implies that $f$ is borel. 
    We can induct on $n$ from the base case of $n=1$ where the proposition is true due to continuity in each variable coinciding with continuity (which is stronger than borel measurability), hence proving the proposition. 
\end{solution}
\begin{problem}\label{prob:2.12}
Prove Proposition 2.20. (See Proposition 0.20, where a special case is proved.)
\end{problem}
\begin{solution}
    If $f^{-1}(\{\infty\})$ is not null then the integral will trivially be infinite and thus it must be a null set. 
    Now, consider the sets $H_n=f^{-1}(1/n,\infty]$ and we find that $\{x:f(x)>0\}=\cup H_n$. 
    Here, we see that $f\geq n^{-1}\chi_{H_n}$ for all $n\geq 1$ and thus $\int f\geq n^{-1}\mu(H_n)$ forces all the $H_n$ to have finite measure. 
\end{solution}
\begin{problem}\label{prob:2.13}
Suppose $\{f_n\} \subset L^+$, $f_n \to f$ pointwise, and $\int f = \lim \int f_n < \infty$. Then $\int_E f = \lim \int_E f_n$ for all $E \in \mM$. However, this need not be true if $\int f = \lim \int f_n = \infty$.
\end{problem}
\begin{solution}
    For any $E\in\mM$ we can see that $f_n\chi_E\longrightarrow f\chi_E$ pointwise as well and thus, by fatou's lemma and the additivity of the integral we have, 
    $$\int f\chi_E\leq\liminf\int f_n\chi_E\text{ and }\int f\chi_{X\setminus E}\leq\liminf\int f_n\chi_{X\setminus E}$$
    $$\implies\int f=\int f\chi_E+\int f\chi_{X\setminus E}\leq\liminf\int f_n\chi E+\liminf\int f\chi_{X\setminus E}$$ 
    Using the fact that the limit infimum is subadditive we see that the above is
    $$\leq\liminf\left(\int f_n\chi_E+f_n\chi_{X\setminus E}\right)=\liminf\int f_n=\int f$$ 
    as all the terms involved in this string of inequalities are finite (this follows from the fact that $\int f<\infty$ and that $\int f_n$ is close to $\int f$ eventually) we must have that, 
    $$\int  f\chi_E=\liminf\int f_n\chi_E$$
    Assume wlog $\int f_n$ is always finite, then by the superadditivity of the limit supremum and the equality we developed earlier, we have
    $$\limsup\int f_n\chi_{E}=\limsup\left(\int f_n-\int f_n\chi_{E^c}\right)\leq\limsup\int f_n+\limsup\left(-\int f_n\chi_{E^c}\right)$$
    $$=\limsup\int f_n-\liminf\int f_n\chi_{E^c}=\int f-\int f\chi_{E^c}=\int f\chi_E$$ 
    $$\limsup\int f_n\chi_E\leq\int f\chi_E=\liminf f_n\chi_E\implies\lim\int f_n\chi_E=\int f\chi_E$$  
    For a counterexample we can consider a function on $\bR$ given by $f_n=n\chi_{(0,1/n)}+\chi_{[1,\infty)}$ which converges pointwise to $\chi_{[1,\infty)}$ and $\int f=\lim\int f_n=\infty$ but if we focus on $(0,1)$ then $\lim f_n\chi_{(0,1)}=1$ whereas $\int f\chi_{(0,1)}=0$. 
    Here the area escaped to infinity in the unit open interval and hence the contradiction. 
\end{solution}
\begin{problem}\label{prob:2.14}
If $f \in L^+$, let $\lambda(E) = \int_E f \, d\mu$ for $E \in \mM$. Then $\lambda$ is a measure on $\mM$, and for any $g \in L^+$, $\int g \, d\lambda = \int gf \, d\mu$. (First suppose that $g$ is simple.)
\end{problem}
\begin{solution}
    For any disjoint collection $\{E_n\}_{n\in\bN}\subseteq\mM$ with union $E$ then by the sigma additivity of the integral we have, 
    $$\lambda(E)=\int f\chi_Ed\mu=\int\sum_{n\in\bN}f\chi_{E_n}d\mu=\sum_{n\in\bN}\int f\chi_{E_n}d\mu=\sum_{n\in\bN}\lambda(E_n)$$ 
    Now, suppose $\phi=\sum a_j\chi_{E_j}$ is a simple function, then we see that 
    $$\int\phi d\lambda=\sum a_j\lambda(E_j)=\sum a_j\int f\chi_{E_j}d\mu=\int\sum a_jf\chi_{E_j}d\mu=\int f\phi d\mu$$ 
    Now, if $g\in L^+,$ we can find a sequence of simple functions $\{\phi_n\}$ that increase to $g$, for these we also see that $f\phi_n$ iincreases to $fg$ and by the monotone convergence theorem and the fact that $\lambda$ and $\mu$ are defined on the same sigma algebra $\mM$ we have that, 
    $$\int fg d\mu=\int\lim f\phi_n d\mu\longleftarrow\int f\phi_n d\mu=\int\phi_n\lambda\longrightarrow \int\lim\phi_n\lambda=\int g\lambda$$ 
    where we used the the monotone convergence theorem on both sides.
\end{solution}
\begin{problem}\label{prob:2.15}
If $\{f_n\} \subset L^+$, $f_n$ decreases pointwise to $f$, and $\int f_1 < \infty$, then $\int f = \lim \int f_n$.
\end{problem}
\begin{solution}
    Clearly we have $\int f\leq\lim\int f_n<\infty$. 
    For the other direction, consider the measurable set $E=\{f>0\}$. 
    Fix $\alpha>1$ and define $F_n:=\{x\in E:f_n(x)\leq\alpha f(x)\}$, then $\{F_n\}$ is an increasing sequence of measurable sets and $\cup F_n=E$. 
    From the definiton we see that $\int_{F_n}f_n\leq\alpha\int_{F_n}f$ and $\int_E f_n-\int_{F_n}f_n=\int_{E\setminus F_n}f_n\leq\int_{E\setminus F_n}f_1$. 
    As $A\mapsto\int_A f_1$ is a measure on $\mM$ and $E\setminus F_n$ is a decreasing sequence with $\int_{E\setminus F_1}f_1\leq\int f_1<\infty$ we see that 
    $$\lim_{n\to\infty}\int_{F\setminus F_n}f_1=\int_{\cap E\setminus F_n}f_1=\int_{E\setminus E}f_1=0$$
    this implies that $\lim\int_{F_n}f_n=\lim\int_E f_n$. 
    As $A\mapsto\int_Af$ is also a measure, we have 
    $$\lim\int_{F_n}f_n=\lim\int_E f_n\implies\lim\int_Ef_n\leq\lim\alpha\int_{F_n} f=\alpha\lim\int_{F_n}f=\alpha\int_{\cup F_n}f=\int_Ef$$ 
    As this is true for any $\alpha>1$ we have $\lim\int_Ef_n\leq\int_Ef$. 
    Now, consider $E^c$ on which $f$ is zero and thus $f_n$ decreases to $0$ here. 
    Let $G_n:=\{x\in E^c:f_n(x)\leq f_1(x)/n\}$ see that $\{G_n\}$ is an increasing sequence of measurable sets with $\cup G_n=E^c$. 
    It can also be shown that $\lim\int_{G_n}f_n=\lim\int_{E^c}f_n$ using similar arguements as we did when working over $E$ and from the definition we see that $\int_{G_n}f_n\leq n^{-1}\int_{G_n}f_1\leq n^{-1}\int f_1\to 0$ and thus, $\lim\int_{E^c}f_n=0=\int_{E^c}f$. 
    Combining these two deductions, namely $\lim\int_Ef_n\leq\int_Ef$ and $\lim\int_{E^c}f_n=0=\int_{E^c}f$ we see that $\lim\int f_n\leq\int f$ as the integral is additive. 
    Having shown both directions of the inequality we must have $\int f=\lim\int f_n$. 
    
\end{solution}
\begin{problem}\label{prob:2.16}
If $f \in L^+$ and $\int f < \infty$, for every $\epsilon > 0$ there exists $E \in \mM$ such that $\mu(E) < \infty$ and $\int_E f > (\int f) - \epsilon$.
\end{problem}
\begin{solution}
From \hyperref[prob:2.12]{Problem 2.12} we see that the set $E=\{x:f(x)>0\}$ is sigma finite so we write it as a disjoint(standard technique) union of countably many sets of finite measure as $E=\cup F_n$, now from the additivity of the integral we have, 
$$\int f=\int f\chi_E+\underbrace{\int f\chi_{E^c}}_{=0}=\int_E f=\sum_{n\in\bN}\int_{F_n}f=\lim_{n}\sum_{k=1}^n\int_{F_n}f=\lim_n\int_{\cup_{k\leq n}F_k}f$$ 
Now, for any $\epsilon>0$ we can find a $N$ such that $\int f<\epsilon+\int_{\cup_{k\leq N}F_k}f$ and as $\mu(\cup_{k\leq N}F_k)=\sum_{k\leq N}\mu(F_k)<\infty$ we are done. 
\end{solution}
\begin{problem}\label{prob:2.17}
Assume Fatou's lemma and deduce the monotone convergence theorem from it.
\end{problem}
\begin{solution}
    Fatou's lemma implies that $\lim\int f_n=\liminf\int f_n\geq\int\liminf f_n=\int f$ while the other direction of the inequality is trivial. 
\end{solution}
\begin{problem}\label{prob:2.18}
Fatou's lemma remains valid if the hypothesis that $f_n \in L^+$ is replaced by the hypothesis that $f_n$ is measurable and $f_n \ge -g$ where $g \in L^+ \cap L^1$. What is the analogue of Fatou's lemma for nonpositive functions?
\end{problem}

\begin{problem}\label{prob:2.19}
Suppose $\{f_n\} \subset L^1(\mu)$ and $f_n \to f$ uniformly.
\begin{enumerate}[label=\alph*.]
    \item If $\mu(X) < \infty$, then $f \in L^1(\mu)$ and $\int f_n \to \int f$.
    \item If $\mu(X) = \infty$, the conclusions of (a) can fail. (Find examples on $\bR$ with Lebesgue measure.)
\end{enumerate}
\end{problem}

\begin{problem}\label{prob:2.20}
(A generalized Dominated Convergence Theorem) If $f_n, g_n, f, g \in L^1$, $f_n \to f$ and $g_n \to g$ a.e., $|f_n| \le g_n$, and $\int g_n \to \int g$, then $\int f_n \to \int f$. (Rework the proof of the dominated convergence theorem.)
\end{problem}

\begin{problem}\label{prob:2.21}
Suppose $f_n, f \in L^1$ and $f_n \to f$ a.e. Then $\int |f_n - f| \to 0$ iff $\int |f_n| \to \int |f|$. (Use Exercise 20.)
\end{problem}

\begin{problem}\label{prob:2.22}
Let $\mu$ be counting measure on $\bN$. Interpret Fatou's lemma and the monotone and dominated convergence theorems as statements about infinite series.
\end{problem}

\begin{problem}\label{prob:2.23}
Given a bounded function $f: [a, b] \to \bR$, let
\[ H(x) = \lim_{\delta \to 0} \sup_{|y-x|\le\delta} f(y), \quad h(x) = \lim_{\delta \to 0} \inf_{|y-x|\le\delta} f(y). \]
Prove Theorem 2.28b by establishing the following lemmas:
\begin{enumerate}[label=\alph*.]
    \item $H(x) = h(x)$ iff $f$ is continuous at $x$.
    \item In the notation of the proof of Theorem 2.28a, $H = G$ a.e. and $h = g$ a.e. Hence $H$ and $h$ are Lebesgue measurable, and $\int_{[a, b]} H \, dm = \overline{I}_a^b(f)$ and $\int_{[a, b]} h \, dm = \underline{I}_a^b(f)$.
\end{enumerate}
\end{problem}

\begin{problem}\label{prob:2.24}
Let $(X, \mM, \mu)$ be a measure space with $\mu(X) < \infty$, and let $(X, \overline{\mM}, \overline{\mu})$ be its completion. Suppose $f: X \to \bR$ is bounded. Then $f$ is $\overline{\mM}$-measurable (and hence in $L^1(\overline{\mu})$) iff there exist sequences $\{\phi_n\}$ and $\{\psi_n\}$ of $\mM$-measurable simple functions such that $\phi_n \le \psi_n$ and $\int (\psi_n - \phi_n) \, d\mu < n^{-1}$. In this case, $\lim \int \phi_n \, d\mu = \lim \int \psi_n \, d\mu = \int f \, d\overline{\mu}$.
\end{problem}

\begin{problem}\label{prob:2.25}
Let $f(x) = x^{-1/2}$ if $0 < x < 1, f(x) = 0$ otherwise. Let $\{r_n\}_1^\infty$ be an enumeration of the rationals, and set $g(x) = \sum_1^\infty 2^{-n} f(x - r_n)$.
\begin{enumerate}[label=\alph*.]
    \item $g \in L^1(m)$, and in particular $g < \infty$ a.e.
    \item $g$ is discontinuous at every point and unbounded on every interval, and it remains so after any modification on a Lebesgue null set.
    \item $g^2 < \infty$ a.e., but $g^2$ is not integrable on any interval.
\end{enumerate}
\end{problem}

\begin{problem}\label{prob:2.26}
If $f \in L^1(m)$ and $F(x) = \int_{-\infty}^x f(t) \, dt$, then $F$ is continuous on $\bR$.
\end{problem}

\begin{problem}\label{prob:2.27}
Let $f_n(x) = ae^{-nax} - be^{-nbx}$ where $0 < a < b$.
\begin{enumerate}[label=\alph*.]
    \item $\sum_1^\infty \int_0^\infty |f_n(x)| \, dx = \infty$.
    \item $\sum_1^\infty \int_0^\infty f_n(x) \, dx = 0$.
    \item $\sum_1^\infty f_n \in L^1([0, \infty), m)$, and $\int_0^\infty \sum_1^\infty f_n(x) \, dx = \log(b/a)$.
\end{enumerate}
\end{problem}

\begin{problem}\label{prob:2.28}
Compute the following limits and justify the calculations:
\begin{enumerate}[label=\alph*.]
    \item $\lim_{n \to \infty} \int_0^\infty (1 + (x/n))^{-n} \sin(x/n) \, dx$.
    \item $\lim_{n \to \infty} \int_0^1 (1 + nx^2)(1 + x^2)^{-n} \, dx$.
    \item $\lim_{n \to \infty} \int_0^\infty n \sin(x/n) [x(1 + x^2)]^{-1} \, dx$.
    \item $\lim_{n \to \infty} \int_a^\infty n(1 + n^2x^2)^{-1} \, dx$. (The answer depends on whether $a > 0$, $a = 0$, or $a < 0$. How does this accord with the various convergence theorems?)
\end{enumerate}
\end{problem}

\begin{problem}\label{prob:2.29}
Show that $\int_0^\infty x^n e^{-x} \, dx = n!$ by differentiating the equation $\int_0^\infty e^{-tx} \, dx = 1/t$. Similarly, show that $\int_{-\infty}^\infty e^{-x^2} e^{2tx} \, dx = (\pi)^{1/2} e^{t^2}$ by differentiating the equation $\int_{-\infty}^\infty e^{-tx^2} \, dx = \sqrt{\pi/t}$ (see Proposition 2.53).
\end{problem}

\begin{problem}\label{prob:2.30}
Show that $\lim_{k \to \infty} \int_0^k x^n (1 - k^{-1}x)^k \, dx = n!$.
\end{problem}

\begin{problem}\label{prob:2.31}
Derive the following formulas by expanding part of the integrand into an infinite series and justifying the term-by-term integration. Exercise 29 may be useful. (Note: In (d) and (e), term-by-term integration works, and the resulting series converges, only for $a > 1$, but the formulas as stated are actually valid for all $a > 0$.)
\begin{enumerate}[label=\alph*.]
    \item For $a > 0$, $\int_0^\infty e^{-x^2} \cos ax \, dx = \sqrt{\pi}e^{-a^2/4}$.
    \item For $a > -1$, $\int_0^1 x^a (1 - x)^{-1} \log x \, dx = \sum_1^\infty (a + k)^{-2}$.
    \item For $a > 1$, $\int_0^\infty x^{a-1} (e^x - 1)^{-1} \, dx = \Gamma(a) \zeta(a)$, where $\zeta(a) = \sum_1^\infty n^{-a}$.
    \item For $a > 1$, $\int_0^\infty e^{-ax} x^{-1} \sin x \, dx = \arctan(a^{-1})$.
    \item For $a > 1$, $\int_0^\infty e^{-ax} J_0(x) \, dx = (a^2 + 1)^{-1/2}$, where $J_0(x) = \sum_0^\infty (-1)^n x^{2n} / 4^n(n!)^2$ is the Bessel function of order zero.
\end{enumerate}
\end{problem}

\begin{problem}\label{prob:2.32}
Suppose $\mu(X) < \infty$. If $f$ and $g$ are complex-valued measurable functions on $X$, define
\[ \rho(f, g) = \int \frac{|f - g|}{1 + |f - g|} \, d\mu. \]
Then $\rho$ is a metric on the space of measurable functions if we identify functions that are equal a.e., and $f_n \to f$ with respect to this metric iff $f_n \to f$ in measure.
\end{problem}

\begin{problem}\label{prob:2.33}
If $f_n \ge 0$ and $f_n \to f$ in measure, then $\int f \le \lim \inf \int f_n$.
\end{problem}

\begin{problem}\label{prob:2.34}
Suppose $|f_n| \le g \in L^1$ and $f_n \to f$ in measure.
\begin{enumerate}[label=\alph*.]
    \item $\int f = \lim \int f_n$.
    \item $f_n \to f$ in $L^1$.
\end{enumerate}
\end{problem}

\begin{problem}\label{prob:2.35}
$f_n \to f$ in measure iff for every $\epsilon > 0$ there exists $N \in \bN$ such that $\mu(\{x : |f_n(x) - f(x)| \ge \epsilon\}) < \epsilon$ for $n \ge N$.
\end{problem}

\begin{problem}\label{prob:2.36}
If $\mu(E_n) < \infty$ for $n \in \bN$ and $\chi_{E_n} \to f$ in $L^1$, then $f$ is (a.e. equal to) the characteristic function of a measurable set.
\end{problem}

\begin{problem}\label{prob:2.37}
If $f_n \to f$ in measure and $g_n \to g$ in measure, then $f_n + g_n \to f + g$ in measure, and $f_ng_n \to fg$ in measure if $\mu(X) < \infty$, but not necessarily if $\mu(X) = \infty$.
\end{problem}

\begin{problem}\label{prob:2.38}
If $f_n \to f$ a.e. and $\mu(X) < \infty$, then $f_n \to f$ in measure.
\end{problem}

\begin{problem}\label{prob:2.39}
If $f_n \to f$ almost uniformly, then $f_n \to f$ a.e. and $f_n \to f$ in measure.
\end{problem}

\begin{problem}\label{prob:2.40}
In Egoroff's theorem, the hypothesis ``$\mu(X) < \infty$'' can be replaced by ``$|f_n| \le g$ for all $n$, where $g \in L^1(\mu)$.''
\end{problem}

\begin{problem}\label{prob:2.41}
If $\mu$ is $\sigma$-finite and $f_n \to f$ a.e., there exist measurable $E_1, E_2, \dots \subset X$ such that $\mu( (\bigcup_1^\infty E_j)^c ) = 0$ and $f_n \to f$ uniformly on each $E_j$.
\end{problem}

\begin{problem}\label{prob:2.42}
Let $\mu$ be counting measure on $\bN$. Then $f_n \to f$ in measure iff $f_n \to f$ uniformly.
\end{problem}

\begin{problem}\label{prob:2.43}
Suppose that $\mu(X) < \infty$ and $f: X \times [0, 1] \to \bC$ is a function such that $f(\cdot, y)$ is measurable for each $y \in [0, 1]$ and $f(x, \cdot)$ is continuous for each $x \in X$.
\begin{enumerate}[label=\alph*.]
    \item If $0 < \epsilon, \delta < 1$ then $E_{\epsilon, \delta} = \{x : |f(x, y) - f(x, 0)| \le \epsilon \text{ for all } y < \delta\}$ is measurable.
    \item For any $\epsilon > 0$ there is a set $E \subset X$ such that $\mu(E) < \epsilon$ and $f(\cdot, y) \to f(\cdot, 0)$ uniformly on $E^c$ as $y \to 0$.
\end{enumerate}
\end{problem}

\begin{problem}\label{prob:2.44}
(Lusin's Theorem) If $f: [a, b] \to \bC$ is Lebesgue measurable and $\epsilon > 0$, there is a compact set $E \subset [a, b]$ such that $m(E^c) < \epsilon$ and $f|_E$ is continuous. (Use Egoroff's theorem and Theorem 2.26.)
\end{problem}

\begin{problem}\label{prob:2.45}
If $(X_j, \mM_j)$ is a measurable space for $j = 1, 2, 3$, then $\bigotimes_1^3 \mM_j = (\mM_1 \otimes \mM_2) \otimes \mM_3$. Moreover, if $\mu_j$ is a $\sigma$-finite measure on $(X_j, \mM_j)$, then $\mu_1 \times \mu_2 \times \mu_3 = (\mu_1 \times \mu_2) \times \mu_3$.
\end{problem}

\begin{problem}\label{prob:2.46}
Let $X = Y = [0, 1]$, $\mM = \mN = \mathcal{B}_{[0, 1]}$, $\mu = \text{Lebesgue measure}$, and $\nu = \text{counting measure}$. If $D = \{(x, x) : x \in [0, 1]\}$ is the diagonal in $X \times Y$, then $\iint \chi_D \, d\mu \, d\nu$, $\iint \chi_D \, d\nu \, d\mu$, and $\int \chi_D \, d(\mu \times \nu)$ are all unequal. (To compute $\int \chi_D \, d(\mu \times \nu)$, go back to the definition of $\mu \times \nu$.)
\end{problem}

\begin{problem}\label{prob:2.47}
Let $X = Y$ be an uncountable linearly ordered set such that for each $x \in X$, $\{y \in X : y < x\}$ is countable. (Example: the set of countable ordinals.) Let $\mM = \mN$ be the $\sigma$-algebra of countable or co-countable sets, and let $\mu = \nu$ be defined on $\mM$ by $\mu(A) = 0$ if $A$ is countable and $\mu(A) = 1$ if $A$ is co-countable. Let $E = \{(x, y) \in X \times X : y < x\}$. Then $E_x$ and $E^y$ are measurable for all $x, y$, but $\iint \chi_E \, d\mu \, d\nu$ and $\iint \chi_E \, d\nu \, d\mu$ exist but are not equal. (If one believes in the continuum hypothesis, one can take $X = [0, 1]$ [with a nonstandard ordering] and thus obtain a set $E \subset [0, 1]^2$ such that $E_x$ is countable and $E^y$ is co-countable [in particular, Borel] for all $x, y$, but $E$ is not Lebesgue measurable.)
\end{problem}

\begin{problem}\label{prob:2.48}
Let $X = Y = \bN, \mM = \mN = \mathcal{P}(\bN), \mu = \nu = \text{counting measure}$. Define $f(m, n) = 1$ if $m = n, f(m, n) = -1$ if $m = n + 1$, and $f(m, n) = 0$ otherwise. Then $\int f \, d\mu \, d\nu$ and $\int f \, d\nu \, d\mu$ exist and are unequal.
\end{problem}

\begin{problem}\label{prob:2.49}
Prove Theorem 2.39 by using Theorem 2.37 and Proposition 2.12 together with the following lemmas.
\begin{enumerate}[label=\alph*.]
    \item If $E \in \mM \times \mN$ and $\mu \times \nu(E) = 0$, then $\nu(E_x) = 0$ for a.e. $x$ and $\mu(E^y) = 0$ for a.e. $y$.
    \item If $f$ is $\mM$-measurable and $f = 0$ $\lambda$-a.e., then $f_x$ and $f^y$ are integrable for a.e. $x$ and $y$, and $\int f_x \, d\nu = \int f^y \, d\mu = 0$ for a.e. $x$ and $y$. (Here the completeness of $\mu$ and $\nu$ is needed.)
\end{enumerate}
\end{problem}

\begin{problem}\label{prob:2.50}
Suppose $(X, \mM, \mu)$ is a $\sigma$-finite measure space and $f \in L^+(X)$. Let
\[ G_f = \{(x, y) \in X \times [0, \infty] : y \le f(x)\}. \]
Then $G_f$ is $\mM \times \mathcal{B}_{\bR}$-measurable and $\mu \times m(G_f) = \int f \, d\mu$; the same is also true if the inequality $y \le f(x)$ in the definition of $G_f$ is replaced by $y < f(x)$. (To show measurability of $G_f$, note that the map $(x, y) \mapsto f(x) - y$ is the composition of $(x, y) \mapsto (f(x), y)$ and $(z, y) \mapsto z - y$.) This is the definitive statement of the familiar theorem from calculus, ``the integral of a function is the area under its graph.''
\end{problem}

\begin{problem}\label{prob:2.51}
Let $(X, \mM, \mu)$ and $(Y, \mN, \nu)$ be arbitrary measure spaces (not necessarily $\sigma$-finite).
\begin{enumerate}[label=\alph*.]
    \item If $f: X \to \bC$ is $\mM$-measurable, $g: Y \to \bC$ is $\mN$-measurable, and $h(x, y) = f(x)g(y)$, then $h$ is $\mM \otimes \mN$-measurable.
    \item If $f \in L^1(\mu)$ and $g \in L^1(\nu)$, then $h \in L^1(\mu \times \nu)$ and $\int h \, d(\mu \times \nu) = (\int f \, d\mu)(\int g \, d\nu)$.
\end{enumerate}
\end{problem}

\begin{problem}\label{prob:2.52}
The Fubini-Tonelli theorem is valid when $(X, \mM, \mu)$ is an arbitrary measure space and $Y$ is a countable set, $\mN = \mathcal{P}(Y)$, and $\nu$ is counting measure on $Y$. (Cf. Theorems 2.15 and 2.25.)
\end{problem}

\begin{problem}\label{prob:2.53}
Fill in the details of the proof of Theorem 2.41.
\end{problem}

\begin{problem}\label{prob:2.54}
How much of Theorem 2.44 remains valid if $T$ is not invertible?
\end{problem}

\begin{problem}\label{prob:2.55}
Let $E = [0, 1] \times [0, 1]$. Investigate the existence and equality of $\int_E f \, dm^2$, $\int_0^1 \int_0^1 f(x, y) \, dx \, dy$, and $\int_0^1 \int_0^1 f(x, y) \, dy \, dx$ for the following $f$:
\begin{enumerate}[label=\alph*.]
    \item $f(x, y) = (x^2 - y^2)(x^2 + y^2)^{-2}$.
    \item $f(x, y) = (1 - xy)^{-a}$ ($a > 0$).
    \item $f(x, y) = (x - \frac{1}{2})^{-3}$ if $0 < y < |x - \frac{1}{2}|$, $f(x, y) = 0$ otherwise.
\end{enumerate}
\end{problem}

\begin{problem}\label{prob:2.56}
If $f$ is Lebesgue integrable on $(0, a)$ and $g(x) = \int_x^a t^{-1} f(t) \, dt$, then $g$ is integrable on $(0, a)$ and $\int_0^a g(x) \, dx = \int_0^a f(x) \, dx$.
\end{problem}

\begin{problem}\label{prob:2.57}
Show that $\int_0^\infty e^{-sx} x^{-1} \sin x \, dx = \arctan(s^{-1})$ for $s > 0$ by integrating $e^{-sxy} \sin x$ with respect to $x$ and $y$. (It may be useful to recall that $\tan(\frac{\pi}{2} - \theta) = (\tan \theta)^{-1}$. Cf. Exercise 31d.)
\end{problem}

\begin{problem}\label{prob:2.58}
Show that $\int e^{-sx} x^{-1} \sin^2 x \, dx = \frac{1}{4} \log(1 + 4s^{-2})$ for $s > 0$ by integrating $e^{-sx} \sin 2xy$ with respect to $x$ and $y$.
\end{problem}

\begin{problem}\label{prob:2.59}
Let $f(x) = x^{-1} \sin x$.
\begin{enumerate}[label=\alph*.]
    \item Show that $\int_0^\infty |f(x)| \, dx = \infty$.
    \item Show that $\lim_{b \to \infty} \int_0^b f(x) \, dx = \frac{1}{2}\pi$ by integrating $e^{-xy} \sin x$ with respect to $x$ and $y$. (In view of part (a), some care is needed in passing to the limit as $b \to \infty$.)
\end{enumerate}
\end{problem}

\begin{problem}\label{prob:2.60}
$\Gamma(x)\Gamma(y)/\Gamma(x+y) = \int_0^1 t^{x-1}(1-t)^{y-1} \, dt$ for $x, y > 0$. (Recall that $\Gamma$ was defined in $\S 2.3$. Write $\Gamma(x)\Gamma(y)$ as a double integral and use the argument of the exponential as a new variable of integration.)
\end{problem}

\begin{problem}\label{prob:2.61}
If $f$ is continuous on $[0, \infty)$, for $\alpha > 0$ and $x \ge 0$ let
\[ I_\alpha f(x) = \frac{1}{\Gamma(\alpha)} \int_0^x (x - t)^{\alpha-1} f(t) \, dt. \]
$I_\alpha f$ is called the $\alpha$th \textbf{fractional integral} of $f$.
\begin{enumerate}[label=\alph*.]
    \item $I_{\alpha+\beta} f = I_\alpha (I_\beta f)$ for all $\alpha, \beta > 0$. (Use Exercise 60.)
    \item If $n \in \bN$, $I_n f$ is an $n$th-order antiderivative of $f$.
\end{enumerate}
\end{problem}

\begin{problem}\label{prob:2.62}
The measure $\sigma$ on $S^{n-1}$ is invariant under rotations.
\end{problem}

\begin{problem}\label{prob:2.63}
The technique used to prove Proposition 2.54 can also be used to integrate any polynomial over $S^{n-1}$. In fact, suppose $f(x) = \prod_1^n x_j^{\alpha_j}$ ($\alpha_j \in \bN \cup \{0\}$) is a monomial. Then $\int f \, d\sigma = 0$ if any $\alpha_j$ is odd, and if all $\alpha_j$'s are even,
\[ \int f \, d\sigma = \frac{2\Gamma(\beta_1) \dots \Gamma(\beta_n)}{\Gamma(\beta_1 + \dots + \beta_n)} \quad \text{where } \beta_j = \frac{\alpha_j + 1}{2}. \]
\end{problem}

\begin{problem}\label{prob:2.64}
For which real values of $a$ and $b$ is $|x|^a |\log |x||^b$ integrable over $\{x \in \bR^n : |x| < \frac{1}{2}\}$? Over $\{x \in \bR^n : |x| > 2\}$?
\end{problem}

\begin{problem}\label{prob:2.65}
Define $G : \bR^n \to \bR^n$ by $G(r, \phi_1, \dots, \phi_{n-2}, \theta) = (x_1, \dots, x_n)$ where
\begin{align*}
x_1 &= r \cos \phi_1, \quad x_2 = r \sin \phi_1 \cos \phi_2, \quad x_3 = r \sin \phi_1 \sin \phi_2 \cos \phi_3, \dots, \\
x_{n-1} &= r \sin \phi_1 \dots \sin \phi_{n-2} \cos \theta, \quad x_n = r \sin \phi_1 \dots \sin \phi_{n-2} \sin \theta.
\end{align*}
\begin{enumerate}[label=\alph*.]
    \item $G$ maps $\bR^n$ onto $\bR^n$, and $|G(r, \phi_1, \dots, \theta)| = |r|$.
    \item $\det D_G(r, \phi_1, \dots, \theta) = r^{n-1} \sin^{n-2} \phi_1 \sin^{n-3} \phi_2 \dots \sin \phi_{n-2}$.
    \item Let $\Omega = (0, \infty) \times (0, \pi)^{n-2} \times (0, 2\pi)$. Then $G|_\Omega$ is a diffeomorphism and $m(\bR^n \setminus G(\Omega)) = 0$.
    \item Let $F(\phi_1, \dots, \phi_{n-2}, \theta) = G(1, \phi_1, \dots, \phi_{n-2}, \theta)$ and $\Omega' = (0, \pi)^{n-2} \times (0, 2\pi)$. Then $(F|\Omega')^{-1}$ defines a coordinate system on $S^{n-1}$ except on a $\sigma$-null set, and the measure $\sigma$ is given in these coordinates by
    \[ d\sigma(\phi_1, \dots, \phi_{n-2}, \theta) = \sin^{n-2} \phi_1 \sin^{n-3} \phi_2 \dots \sin \phi_{n-2} \, d\phi_1 \dots d\phi_{n-2} \, d\theta. \]
\end{enumerate}
\end{problem}
\end{document}